\documentclass[a4paper]{article}
    \usepackage{fullpage}
    \usepackage{amsmath}
    \usepackage{amssymb}
    \usepackage{textcomp}
    \usepackage[utf8]{inputenc}
    \usepackage[T1]{fontenc}
    \usepackage[top=3cm, bottom=0.1cm, left=1.6cm, right=1.5cm]{geometry}
    \usepackage[hidelinks]{hyperref}
    \textheight=10in
    \pagestyle{empty}
    \usepackage{enumitem}
    \raggedright

\def\bull{\vrule height 0.8ex width .7ex depth -.1ex }

% DEFINITIONS FOR RESUME %%%%%%%%%%%%%%%%%%%%%%%

\urlstyle{same}

\newcommand{\area} [2] {
    \vspace*{-9pt}
    \begin{verse}
        \textbf{#1}   #2
    \end{verse}
}

\newcommand{\lineunder} {
    \vspace*{-8pt} \\
    \hspace*{-10pt} \hrulefill \\
}

\newcommand{\header} [1] {
    {\hspace*{-10pt}\vspace*{6pt} \textsc{#1}}
    \vspace*{-6pt} \lineunder
}

\newcommand{\employer} [3] {
    { \textbf{#1} (#2)\\ \underline{\textbf{\emph{#3}}}\\  }
}

\newcommand{\contact} [3] {
    \vspace*{-10pt}
    \begin{center}
        {\Huge \scshape {#1}}\\
        #2 \\ #3
    \end{center}
    \vspace*{-8pt}
}

\newenvironment{degrees}{
    \small
}

\newcommand{\school} [4] {
    \textbf{#1} #2 $\bullet$ #3\\
    #4 \\
}

\newenvironment{myitemize}
{   \small
    \vspace{-2pt}
    \begin{itemize}
    \setlength{\itemsep}{0pt}
    \setlength{\parskip}{0pt}
    \setlength{\parsep}{0pt}     }
{ \end{itemize}                  } 

\newenvironment{singleitem}
{   \small
    \vspace{0pt}
    \begin{itemize}
    \setlength{\itemsep}{0pt}
    \setlength{\parskip}{0pt}
    \setlength{\parsep}{0pt}   }
{\end{itemize} \vspace{1pt}	}
% END RESUME DEFINITIONS %%%%%%%%%%%%%%%%%%%%%%%

\begin{document}
\vspace*{-70pt}

%==== Profile ====%
\begin{center}
	{\href{https://www.linkedin.com/in/nicholaslambourne/}{\Huge \scshape {Nicholas Lambourne}}}\\
	Brisbane QLD, Australia $\cdot$ \href{mailto:nick@ndl.im}{nick@ndl.im} $\cdot$ \href{tel:61407980730}{+61 407 980 730} $\cdot$ \href{https://ndl.im}{ndl.im}\\
\end{center}

\vspace{-2mm}

%==== Education ====%
\header{Education}
\textbf{\href{https://www.unsw.edu.au/}{The University of New South Wales}}\hfill Sydney, NSW, Australia\\
\begin{degrees}
	Bachelor of Science (Honours; Computer Science \& Engineering) \hfill Feb 2021 - Jul 2022
\end{degrees}
\textbf{\href{https://uq.edu.au/}{The University of Queensland}}\hfill Brisbane, QLD, Australia\\
\begin {degrees}
Bachelor of Science (Computer Science) \hfill Graduated 2020\\
%\begin{itemize}[noitemsep,topsep=0pt]
%	\setlength{\itemsep}{0pt}
%    \setlength{\parskip}{0pt}
%    \setlength{\parsep}{0pt}
%	%\item Exchange Studies (Computer Science) - The University of Connecticut \hfill Jan 2018 - May 2018
%%	\item Thesis: \textit{Software-Level Implementation of Quantum Finite Automata}
%\end{itemize}
Bachelor of Commerce (Finance) \& Bachelor of Science (Psychology) \hfill Graduated 2015\\
\end{degrees}

%\textbf{The University of Connecticut}\hfill Storrs, CT, United States\\
%\begin {degrees}
%Exchange Studies (Computer Science) \hfill Jan 2018 - May 2018\\
%\end{degrees}

%\vspace{1mm}

%==== Experience ====%
\vspace{1mm}
\header{Experience}

\textbf{\href{https://www.canva.com/}{Canva}} \hfill Sydney, NSW\\
\textit{Software Engineer} \hfill Feb 2021 - Present\\
\begin{myitemize} \itemsep 0.5mm
	\item Newly attached to the Platform \& Tools team of the Data Engineering group, working to maintain Canva's data lake, associated workflows, and the hosted notebooks platform that facilitates ML development within the company.
\end{myitemize}
\vspace{-1mm}
\textit{Software Engineering Intern} \hfill Dec 2019 - Feb 2020\\
\begin{myitemize} \itemsep 0.5mm
	\item As part of the infrastructure team, built a shell script transpiler framework and reference generator targeting Python 3.
	\item Created an interactive playground, using Python, Flask and React, allowing for real-time transpilation in the browser.
\end{myitemize}

%\vspace{-0.5mm}

\textbf{\href{https://www.itee.uq.edu.au/}{The University of Queensland - School of IT \& Electrical Engineering}} \hfill Brisbane, QLD\\
\textit{Research Assistant} \hfill Jun 2018 - Present\\
\begin{myitemize} \itemsep 0.5mm
	\item Contributed to the development of \href{https://github.com/CoEDL/elpis}{Elpis}, abstracting away the complexity of the Kaldi automatic speech recognition (ASR) library. Our team took a CLI toolkit and developed an abstraction layer and GUI using React and Flask.
%	\item Developed resource creation software for the Ngukurr OPIE Social Robot Project (see Hermes, below). 
\end{myitemize}
\vspace{-1mm}
\textit{Teaching Assistant} \hfill Jul 2018 - December 2020\\
\vspace{-0.5mm}
\begin{myitemize} \itemsep 0.5mm
	\item 
		\href{http://bit.ly/comp3001-ecp}
			{\textit{Voyages in Language Technologies} (COMP3001)}: 
			taught computational linguistics and helped prepare course materials.
	\item 
		\href{http://bit.ly/csse2310-ec}
			{\textit{Computer Systems Principles and Programming} (CSSE2310)}: 
			taught students systems programming using C and Unix.
	\item 
		\href{http://bit.ly/csse2002-ecp}
			{\textit{Programming in the Large} (CSSE2002)}: 
			taught Java/OO programming and developed course teaching materials.
%	\item Responsible for teams of first-year engineering students taking the course \textit{Engineering Modelling and Problem Solving} (ENGG1200), in which they utilise industry tools including CREO and Matlab to design, build and test a model UAV.
\end{myitemize}

%\vspace{-0.5mm}

\textbf{\href{https://www.atlassian.com/}{Atlassian}} \hfill Sydney, NSW\\
\textit{Site Reliability Engineering Intern} \hfill Nov 2020 - Feb 2021 \\
\begin{myitemize}
	\item Attached to the Platform SRE team, I am worked on various full stack (PostgresSQL, Java, Spring, TypeScript) projects related to the reporting of service level objective (SLO) performance to internal teams.
\end{myitemize}


\vspace{-0.5mm}

\textbf{\href{https://www.patchdmedical.com/}{Patchd, Inc.}} \hfill San Francisco, CA\\
\textit{Software Engineering Intern} \hfill  Jan 2019 - Feb 2019\\
\begin{myitemize} \itemsep 0.5mm
%	\item Patchd is a Y-Combinator backed startup working on the detection of sepsis from vital sign data using machine learning.
	\item Worked on parallelisation strategies for training recurrent neural nets in the detection of sepsis from vital sign data.
\end{myitemize}

%\textbf{ADLED Pty. Ltd. \& SS Signs} \hfill Cleveland, QLD\\
%\textit{Technical Advisor} \hfill Oct 2015 - Dec 2017\\
%\vspace{-0.5mm}
%\begin{myitemize} \itemsep 0.5mm
%	\item Developed software to monitor a network of LED billboards and control the rotation of certain installations (see projects).
%	\item Compiled financial reports and worked with industry groups to lobby local governments on legislative issues.
%\end{myitemize}

\vspace{-0.5mm}
\header{Skills}
\vspace{1mm}
\begin{tabular}{p{0.12\linewidth}p{0.83\linewidth}}
	\textbf{Languages:} & Python, Java, C, TypeScript/JavaScript, SQL, HTML/CSS, \LaTeX \\
%	\small Encountered: & \small MATLAB, Ruby, AVR Assembly, SML, Smalltalk, Prolog\\
%\end{tabular}
%\textbf{Other Skills}
%\begin{tabular}{p{0.15\linewidth}p{0.80\linewidth}}
	\textbf{Tools:} & Web Frameworks (Flask, Django, ExpressJS), SQL DBs (PostgreSQL, MySQL), NoSQL (Mongo), CI/CD (Jenkins, Travis), Cloud Platforms, Containerisation (Docker)
\end{tabular}

\vspace{0mm}

\header{Projects}

\textbf{\href{https://github.com/CoEDL/elpis}{Elpis - Accelerated Linguistic Transcription \hfill github.com/CoEDL/elpis}}
\begin{singleitem}
	\item Part of a team developing an abstraction framework and GUI for speech recognition using the Kaldi ASR library.
\end{singleitem}

\textbf{\href{https://github.com/nicklambourne/slackblocks}{SlackBlocks - Slack API Wrapper \& PyPI Package \hfill github.com/nicklambourne/slackblocks}}
\begin{singleitem}
	\item A Python library, published on PyPI, for back-porting the Slack Blocks API to earlier versions of the Slack client.
\end{singleitem}

\textbf{\href{https://github.com/CoEDL/hermes}{Hermes - Language Resource Creator \hfill github.com/CoEDL/hermes}}
\begin{singleitem}
	\item A cross-platform Python/Qt application for creating language teaching resources from linguistic analysis files.
\end{singleitem}

%\textbf{\href{https://ndl.im}{ndl.im} (\href{https://github.com/nicklambourne/personal}{https://github.com/nicklambourne/personal})}
%\begin{singleitem}
%	\item Personal website built using NodeJS, ExpressJS and MongoDB with an automated build pipeline and Jenkins for CI/CD.
%\end{singleitem}

\textbf{\href{https://github.com/UQComputingSociety/uqcsbot}{UQCSbot - A Python-Based, Extensible Slack Bot \hfill github.com/UQComputingSociety/uqcsbot}}
\begin{singleitem}
	\item Contributed new Python scripts, script overhauls, and code reviews to my university computing society's Slack bot.
\end{singleitem}



%\textbf{LED Billboard Rotation System} \hfill \textbf{(closed source)}
%\begin{singleitem}
%	\item Python system for controlling the rotation of large-format LED billboards including GUI and concurrency features.
%\end{singleitem}

\vspace{0mm}

\header{Volunteering}
\textbf{\href{https://uqcs.org}{University of Queensland Computing Society}} \hfill 2015 - Present\\
\begin{singleitem}
%	\item Elected president of the 500 member society for the 2019 academic year.  An active member since 2014.
	\item As president in 2019 I organised talks and events, secured sponsorships, and managed the club's open source projects.
	\item Oversaw a 30\% increase in membership, twofold increase in number of events held, and a 75\% increase in sponsorship.
%	\item The society won the \textit{Best Faculty Club of the Year} award at the 2019 UQ Union Clubs \& Societies awards night.
	\item Returned to manage the events portfolio as part of the 2020 executive committee, organising more than 50 events.
	\item In 2020 I was named an honorary life member in recognition of extraordinary contributions to the society.
\end{singleitem}
\textbf{\href{https://www.nodegirls.com/}{Node Girls}, \href{https://djangogirls.org/}{Django Girls}, \& \href{https://robogalsbrisbane.org/}{Robogals} (Brisbane Chapters)} \hfill 2016 - 2020\\
\begin{singleitem}
	\item Provided mentorship in web and robotics technologies in an effort to increase female representation in technology.
\end{singleitem}
%\textbf{YOW! Developer Conferences} \hfill 2017 - Present\\
%\begin{singleitem}
%	\item Participated as a student volunteer assisting with conference organisation and preparation.
%\end{singleitem}

\vspace{0mm}

\header{Scholarships, Grants \& Prizes}
\textbf{\href{https://www.eait.uq.edu.au/eait-scholars-program}{UQ Faculty of Engineering, Architecture \& IT Hawken Scholar}}\hfill (2020)\\
\textbf{\href{https://ventures.uq.edu.au/san-fran}{UQ IdeaHub San Francisco Startup Adventure Participant}} \hfill (2019)\\
\textbf{\href{https://employability.uq.edu.au/summer-winter-research}{UQ Summer Research Scholarship}} \hfill (2019)\\
\textbf{\href{https://employability.uq.edu.au/financial-support/employability-grant}{UQ Advantage Grant} (\href{http://www.dynamicsoflanguage.edu.au/education-and-outreach/train-with-us/summer-school-2018/}{CoEDL Summer School})} \hfill (2018)\\
\textbf{\href{https://employability.uq.edu.au/summer-winter-research}{UQ Winter Research Scholarship}} \hfill (2018)\\
%\textbf{University of Connecticut Innovation Quest - Finalist} \hfill The University of Connecticut (2018)\\
% \textbf{UQ Abroad Travel Grant (University of Connecticut)} \hfill (2018)\\
%\textbf{Dean\textquotesingle{}s Commendation for Academic Achievement} \hfill The University of Queensland (2011)\\


\vspace{1.5mm}
\begin{center}
\small \textit{References available upon request}
\end{center}

\end{document}